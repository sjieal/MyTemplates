%%%Document%%%
\documentclass[12pt,a4paper,twoside,UTF8]{ctexart}
%%%Packages%%%
\usepackage{amsmath,amsthm,amssymb,graphicx}%公式定理宏包
\usepackage[bookmarks=true,colorlinks,citecolor=black,linkcolor=black]{hyperref}
\usepackage{fancyhdr}%页眉页脚宏包
\usepackage{tcolorbox}%框线宏包
\usepackage{geometry}%页面宏包
\usepackage{authblk}%作者环境
\usepackage{tikz}
\usepackage{listings}
%%%Page style%%%
\geometry{a4paper,left=25mm,right=20mm,top=25mm,bottom=20mm}%设定页面
\pagestyle{fancy}%fancy 在geometry 后面
\fancyhf{}
\renewcommand{\headrulewidth}{0.5mm}%页眉线大小
\fancyhead[C]{数值分析课程笔记}
\fancyfoot[RO, LE]{\thepage}
\setlength{\headheight}{15pt}
\linespread{1.5}%设定行距

%%%Define Chinese 定理,例%%%
\newtheorem{example}{\indent 例}[section]
\newtheorem{theorem}{定理}

\pagenumbering{Roman}
\title{A Note Template for Sijie Li}
\author{Sijie Li}
\affil{Department of Logistics Management and Engineering, Southeast University}
\date{\today}

\begin{document}
    %\setcounter{page}{0}
    \maketitle

    \newpage
    \begin{titlepage}
    \vspace*{\stretch{1}}
    \begin{center}
      {\huge \bf Template \\[1ex]
                      A Note Template for Sijie Li}                  \\[6.5ex]
      {\large Sijie Li}           \\
      \vspace{15cm}
      \textit{School of Economics and Management, Southeast University}                \\[2ex]
      \vfill
      \today
    \end{center}
    \vspace{\stretch{2}}
    \end{titlepage}

    \thispagestyle{empty}
    \newpage
    \tableofcontents %目录

    \newpage
    \pagenumbering{arabic}
    \section{数值计算误差分析}
    \subsection{误差分类}
    误差按其性质和产生原因, 可分为系统误差, 随机误差和过失误差.

    (1) 系统误差: 又称可测误差, 恒定误差或偏倚. 指测量值的总体均值与真值之间的差别, 是由测量过程中某些恒定因素造成的, 在一定条件下具有重现性, 并不因增加测量次数而减少系统误差, 它的产生可以是方法、仪器、试剂、恒定的操作人员和恒定的环境造成.

    (2) 随机误差: 又称偶然误差或不可测误差. 是由测定过程中各种随机因素的共同作用所造成, 随机误差遵从正态分布规律.

    (3) 过失误差: 又称粗差. 是由测量过程中犯了不应有的错误所造成, 它明显地歪曲测量结果, 因而一经发现必须及时改正.

    \begin{theorem}[截断误差]
    模型的准确解与某种数值方法的准确解之间的误差称为截断误差或方法误差.
    \end{theorem}

    \begin{theorem}[舍入误差]
    用计算机计算, 由于计算机字长有限而在数值运算的每一步所产生的误差称为舍入误差.
    \end{theorem}

    \subsection{绝对误差和相对误差}
    设数$x$是精确值, $x^*$是$x$的一个近似值.

    称$e$为近似值的绝对误差, 称$e_r$为近似值的相对误差:

    \[e=x^*-x \eqno(1.1)\]

    \[e_r=\frac{x^*-x}{x}=\frac{e}{x}\eqno(1.2)\]

    \par
    一般为了方便计算, 我们认为:
    \[e_r \approx \frac{e}{x^*}\eqno(1.3)\]

    称 $\delta$ 为数$x$的近似值 $x^*$ 的绝对误差限, 称 $\delta_r$ 为近似值 $x^*$ 的相对误差限:
    \[|e|=|x^*-x| \leq \delta\eqno(1.4)\]
    \[\left|e_r\right| = \frac{\left|x^* - x\right|}{|x|} \leq \delta_r \eqno (1.5)\]

    \begin{example}[绝对误差限]
    已知准确值 $a = 3.1415926\ldots$ 是一个无限不循环小数, 求截取不同位数后的近似值和误差界.
    \end{example}

    截取一位: \(x_1 = 3\),

    \(e_1 = |a - x_1| \approx 0.14 < \frac{1}{2} \times 10^0\);

    截取三位: \(x_1 = 3.14\),

    \(e_2 = |a - x_2| \approx 0.0016 < \frac{1}{2} \times 10^{-2}\);

    截取五位: \(x_1 = 3\),

    \(e_3 = |a - x_3| \approx 0.0000074 < \frac{1}{2} \times 10^{-4}\).

    $a$四舍五入的近似值的绝对误差都超不过本身末位数的半个单位.

    \subsection{有效数字位数}
    \begin{theorem}[有效数字]
    设近似数$x^*=\pm a_1.a_2 \cdots a_n a_{n+1} \cdots a_m \times 10^p$的绝对误差限是第$n$位数字的半个单位, 即 $|x^*-x| \leq \frac{1}{2} \times 10^{1-n+p}$, 则数$x^*$有$n$位有效数字.
    \end{theorem}

    也可以说, 若误差限是某一位的半个单位, 则从该位起到第一个非零数字的位数, 就是有效数字位数.

    \begin{example}[有效数字位数]
    设$x = \sqrt{3} = 1.7320508 \cdots$. $x_1 = 1.73, x_2 = 1.7321, x_3 = 1.7320$是其近似值, 问它们分别有几位有效数字?
    \end{example}
    解:
    \vspace{-.5cm}
    \begin{align*}
        e_1 & = |\sqrt{3} - x_1| \approx 0.0021 < \frac{1}{2} \times 10^{-2} \quad \quad &3\text{位}\\
        e_2 & = |\sqrt{3} - x_2| \approx 0.00005 = \frac{1}{2} \times 10^{-4} \quad &5\text{位}\\
        e_3 & = |\sqrt{3} - x_3| \approx 0.0021 < \frac{1}{2} \times 10^{-3} \quad &{\color{red}4}\text{位}
    \end{align*}

     四舍五入得到的近似数的有效数字位数 = 末位到第一个非零数字的个数(每一位都是有效数字).

    \begin{tcolorbox}[title = {关系}]
    绝对误差限与有效数字位数的关系: $p$相同的情况下, $n$越多,绝对误差限越小;
    \par{相对误差限与有效数字位数的关系: $n$越多, 相对误差限越小.}
    \end{tcolorbox}

    \begin{theorem}[有效数字与相对误差限]
    设近似数 $x^*=\pm a_1.a_2 \cdots a_n \times 10^m$, 若
    $x^{*}$有$n$位有效数字, 则其相对误差限为$\varepsilon_{\mathrm{r}} \leq \frac{1}{2 a_1} \times 10^{-(n-1)}$.
    \end{theorem}

    \begin{example}[有效数字与相对误差限]
    要使$\sqrt{20}$的近似值的相对误差限不超过0.1\%, 则它的近似值至少应该取几位有效数字?
    \end{example}

    解:
    \[\sqrt{20} = 4.4 \cdots\]

    设至少应取\(n\)位有效数字, 则其相对误差限应该满足:

    \[\epsilon_{r}(x^{*}) \le \frac{1}{2 a_{1}} \times 10^{- n + 1} = \frac{1}{2 \times 4} \times 10^{- n + 1} \le 0.1\%\]

    解得 \(n \ge 4\).

    当某个量的准确值很小或很大时, 相对误差比绝对误差更能反映准确数与近似数的差异. 一个近似值的准确程度, 不仅与绝对误差的大小有关, 而且与准确值本身的大小有关.

    \newpage
    \section{Introduction to TiKZ}
    \LaTeX, 作为学术界 de facto 的排版利器, 其编辑器可谓是百花齐放. VS Code, TeX-Studio, TeXMaker, 甚至在线的 Overleaf, 虽然都可以拿来撰写 \LaTeX 文档, 但是它们的界面都相对臃肿, 编译和预览速度上也不是「业内最快」. 经历了近一个月的英文学术论文撰写, 我发现直接使用被誉为「编辑器之神」的 Vim (实际上我使用的是 NeoVim) 来撰写 \LaTeX, 就我而言最为合适. 使用 NeoVim 编写 \LaTeX 文献, 不仅编译飞快, 预览方便, 还有着几乎无限的可定制性. 唯一的缺点, 可能就是 Vim 自己相对陡峭的学习曲线.

    \subsection{第一个例子}
    我们先从一个最简单的例子开始:画一条直线\footnote{这里的例子是直线或箭线.}.

    \begin{lstlisting}
    \begin{tikzpicture}
      \draw (0,0) - - (1,1);
    \end{tikzpicture}
    \end{lstlisting}

    \begin{tikzpicture}
      \draw (0,0) -- (1,1);
    \end{tikzpicture}

    可以看到, 这段代码的核心是 `\textbackslash draw (0,0) -~- (1,1)', 意思就是从 $(0,0)$ 到 $(1,1)$ 画一条线段.
    
    我们还可以画的稍微复杂一点.

    \begin{tikzpicture}
      \draw [color=blue!50, ->] (0,0) node [left] {$A$} -- node [color=red!70, pos=0.25, above, sloped] {Hello, (10,10)} (4,4) node [right] {$B$};
    \end{tikzpicture}

    可以看到, \textbackslash draw 后面的[ ]中跟的是线的一些设置, color=blue!50表示的是用 50\% 的蓝色, 因为\LaTeX中,\% 用作了注释, 所以这里用!替代, \textendash\textgreater 表示的是线形是一个箭头. 注意到, 在起点坐标, \textendash, 终点坐标后面, 分别加入了一个 node 元素, 起点后面的 node 表示的是加入一个标示, 它在坐标点$(0, 0)$的左边, \textendash~\textendash 后面的 node 采用 70\% 的红色, 在线段的上方0.25的位置, 随线段倾斜, 花括号中是 node 的文字 \{Hello\}, 终点坐标同理. node 经常用于加入一些标注, 这一点我们在后面将会看到.

    \subsection{一些复杂的形状} \label{this}
    在 TikZ 中, 除了画线段之外, 还支持各种复杂的形状, 下面一个例子给出了一些常见的形状.

    \begin{tikzpicture}
      \draw (0,0) circle (2cm); %画一个圆心在原点,半径为2cm的圆

    \end{tikzpicture}

    \underline{Supply Chinese}.
    
    今天上了一天课, \emph{上午项目 Project Management, 下午信息 Management Information Systems}.
    \\*
    Yes, let me have a look about Section~\ref{this} on page ``\pageref{this}''.

    No problem.\\[20mm]
    
    This the line space.
\end{document}
