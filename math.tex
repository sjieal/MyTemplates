%\begin{设置文本类型、引入宏包}
\documentclass[12pt,a4paper]{ctexart}%文本类型,文本类型为中文文章
\usepackage[utf8]{inputenc}%所使用的宏包
\usepackage[T1]{fontenc}
\usepackage{amsmath}%常用的数学宏包
\usepackage{amssymb}
\usepackage{graphicx}
\usepackage{float}
\RequirePackage{caption}
\usepackage{subfigure}
\usepackage{fancyhdr}
\usepackage{listings}
\usepackage{makecell}%表格内换行的宏包
\usepackage{diagbox}
\usepackage{listings}
\usepackage[colorlinks, linkcolor=black, anchorcolor=blue, citecolor=black]{hyperref}
\usepackage{tikz}
\usepackage{pxfonts}
\usepackage{geometry}
\usepackage{setspace}
%\usepackage{mathrsfs}%数学字体宏包
\usepackage{amsfonts}%数学字体宏包
\usepackage{bm}%数学符号加粗
\usepackage{comment}%批量注释的宏包
%\end{设置文本类型、引入宏包}
\usepackage{boxedminipage}
\usepackage{shadow}

%\begin{设置页面尺寸}
\setstretch{1} 
\geometry{a4paper,left=25mm,right=25mm,top=25mm,bottom=25mm}
%\end{设置页面尺寸}

%字号的设置用\zihao{*} 其中*为数字,-4为小四,4为四号,3为三号
%字体的设置用	{\songti 宋体}、{\heiti 黑体}、{\fangsong 仿宋}、{\kaishu 楷书},使用时把字加在花括号内如{\songti 同济大学}
%不使用上述代码时默认宋体小四

%\begin{抄录代码环境lstlisting的设置},不需要comment环境注释掉即可
\usepackage{listings}
\usepackage{color}

\definecolor{dkgreen}{rgb}{0,0.6,0}
\definecolor{gray}{rgb}{0.5,0.5,0.5}
\definecolor{mauve}{rgb}{0.58,0,0.82}

\lstset{frame=tb,
	language=Python,
	aboveskip=3mm,
	belowskip=3mm,
	showstringspaces=false,
	columns=flexible,
	basicstyle={\small\ttfamily},
	numbers=none,
	numberstyle=\tiny\color{gray},
	keywordstyle=\color{blue},
	commentstyle=\color{dkgreen},
	stringstyle=\color{mauve},
	breaklines=true,
	breakatwhitespace=true,
	tabsize=3
}
%\end{抄录代码环境lstlisting的设置}

%\begin{页眉页脚设置}
\pagestyle{fancy}
\fancyhf{} %清除原页眉页脚样式
\renewcommand{\headrulewidth}{0mm} % 设置页眉页脚横线及样式 %页眉线宽,设为0可以去页眉线
\renewcommand{\footrulewidth}{0mm} %页脚线宽,设为0可以去页眉线
\cfoot{\thepage} 
%\end{页眉页脚设置}

%\begin{正文开始}
\begin{document}

\centerline{\zihao{2}{\heiti 数学公式编辑}}

~\\

\noindent {\zihao{4} {\heiti 一、球坐标}}

~\\

\noindent \setlength\sdim{10pt}
\shabox{\parbox{7cm}{$\left\{
\begin{array}{lcl}
x=r\sin\varphi\cos\theta,  &     &{0\leqslant\varphi\leqslant\pi}\\
y=r\sin\varphi\sin\theta,  &     &{0\leqslant\theta\leqslant 2\pi}\\
z=a\cos\varphi.            &     &{}\\
\end{array}
\right.$}}

~\\

\begin{lstlisting}
$\left\{
\begin{array}{lcl}
	x=r\sin\varphi\cos\theta,  &     &{0\leqslant\varphi\leqslant\pi}\\
	y=r\sin\varphi\sin\theta,  &     &{0\leqslant\theta\leqslant 2\pi}\\
	z=a\cos\varphi.            &     &{}\\
\end{array}
\right.$
\end{lstlisting}

~\\

\noindent {\zihao{4}{\heiti 二、雅可比式}}

~\\

\noindent \setlength\sdim{10pt}
\shabox{\parbox{5cm}{$J=\cfrac{\partial (F,G)}{\partial (\mu ,\nu)}=\begin{vmatrix}\cfrac{\partial F}{\partial \mu}&\cfrac{\partial F}{\partial \nu}\\\cfrac{\partial G}{\partial \mu}&\cfrac{\partial G}{\partial \nu}\end{vmatrix}$}}

~\\

\begin{lstlisting}
$J=\cfrac{\partial (F,G)}{\partial (\mu ,\nu)}=\begin{vmatrix}\cfrac{\partial F}{\partial \mu}&\cfrac{\partial F}{\partial \nu}\\\cfrac{\partial G}{\partial \mu}&\cfrac{\partial G}{\partial \nu}\end{vmatrix}$
\end{lstlisting}

~\\

\noindent {\zihao{4}{\heiti 三、拉格朗日乘数法}}

~\\

\noindent \setlength\sdim{10pt}
\shabox{\parbox{5cm}{$\left\{
\begin{array}{l}
f_{x}(x,y)+\lambda{\varphi}_{x}(x,y)=0,\\
f_{y}(x,y)+\lambda{\varphi}_{y}(x,y)=0,\\
\varphi (x,y)=0.
\end{array}\right.$}}

\clearpage

\begin{lstlisting}
$\left\{
\begin{array}{l}
	f_{x}(x,y)+\lambda{\varphi}_{x}(x,y)=0,\\
	f_{y}(x,y)+\lambda{\varphi}_{y}(x,y)=0,\\
	\varphi (x,y)=0.
\end{array}\right.$
\end{lstlisting}

~\\

\noindent {\zihao{4}{\heiti 四、二元函数的泰勒公式}}

~\\

\noindent \setlength\sdim{10pt}
\shabox{\parbox{14cm}{$\begin{aligned}
f(x_{0}+h,y_{0}+h)=&f(x_{0},y_{0})+(h\cfrac{\partial}{\partial x}+k\cfrac{\partial}{\partial y})f(x_{0},y_{0})+\\
&\cfrac{1}{2!}(h\cfrac{\partial}{\partial x}+k\cfrac{\partial }{\partial x})^{2}f(x_{0},y_{0})+\cdots +\cfrac{1}{n!}(h\cfrac{\partial}{\partial x}+k\cfrac{\partial }{\partial x})^{n}f(x_{0},y_{0})+\\
&\cfrac{1}{(n+1)!}(h\cfrac{\partial}{\partial x}+k\cfrac{\partial }{\partial x})^{n+1}f(x_{0}+\theta h,y_{0}+\theta k)\quad (x<\theta <1).
\end{aligned}$}}

~\\

\begin{lstlisting}
$\begin{aligned}
	f(x_{0}+h,y_{0}+h)=&f(x_{0},y_{0})+(h\cfrac{\partial}{\partial x}+k\cfrac{\partial}{\partial y})f(x_{0},y_{0})+\\
	&\cfrac{1}{2!}(h\cfrac{\partial}{\partial x}+k\cfrac{\partial }{\partial x})^{2}f(x_{0},y_{0})+\cdots +\cfrac{1}{n!}(h\cfrac{\partial}{\partial x}+k\cfrac{\partial }{\partial x})^{n}f(x_{0},y_{0})+\\
	&\cfrac{1}{(n+1)!}(h\cfrac{\partial}{\partial x}+k\cfrac{\partial }{\partial x})^{n+1}f(x_{0}+\theta h,y_{0}+\theta k)\quad (x<\theta <1).
\end{aligned}$
\end{lstlisting}

~\\

\noindent {\zihao{4}{\heiti 五、极坐标中的二重积分公式}}

~\\

\noindent \setlength\sdim{10pt}
\shabox{\parbox{12cm}{$\displaystyle{\mathop{\iint}\limits_{D}f(\rho\cos\theta ,\rho\sin\theta)\rho d\rho d\theta
=\int_{\alpha}^{\beta}d\theta\int_{{\varphi}_{1}(\theta)}^{{\varphi}_{2}(\theta)}f(\rho\cos\theta ,\rho\sin\theta)\rho d\rho}$}}

\clearpage

\noindent {\zihao{4}{\heiti 六、球坐标中的三重积分公式}}

~\\

\noindent \setlength\sdim{10pt}
\shabox{\parbox{10cm}{$\displaystyle{\mathop{\iiint}\limits_{D}f(x,y,z)dxdydz=\mathop{\iiint}\limits_{D}F(
r,\varphi ,\theta)r^{2}\sin\varphi drd\varphi d\theta}$}}

~\\

\begin{lstlisting}
$\displaystyle{\mathop{\iiint}\limits_{D}f(x,y,z)dxdydz=\mathop{\iiint}\limits_{D}
F(r,\varphi ,\theta)r^{2}\sin\varphi drd\varphi d\theta}$
\end{lstlisting}

~\\

\noindent {\zihao{4}{\heiti 七、格林公式}}

~\\

\noindent \setlength\sdim{10pt}
\shabox{\parbox{7cm}{$\displaystyle{\mathop{\iint}\limits_{D}(\cfrac{\partial Q}{\partial x}-\cfrac{\partial P}{\partial y})dxdy=\mathop{\oint}_{L}Pdx+Qdy}$}}

~\\

\begin{lstlisting}
$\displaystyle{\mathop{\iint}\limits_{D}(\cfrac{\partial Q}{\partial x}-\cfrac{\partial P}{\partial y})dxdy=\mathop{\oint}_{L}Pdx+Qdy}$
\end{lstlisting}

~\\

\noindent {\zihao{4}{\heiti 八、对面积的曲面积分}}

~\\

\noindent \setlength\sdim{10pt}
\shabox{\parbox{12cm}{$\displaystyle{\mathop{\iint}\limits_{\Sigma}f(x,y,z)dS=\mathop{\iint}\limits_{D_{xy}}f[x,y,z(x,y)]\sqrt{1+z_{x}^{2}(x,y)+z_{y}^{2}(x,y)}dxdy}$}}

~\\

\begin{lstlisting}
$\displaystyle{\mathop{\iint}\limits_{\Sigma}f(x,y,z)dS=\mathop{\iint}\limits
_{D_{xy}}f[x,y,z(x,y)]\sqrt{1+z_{x}^{2}(x,y)+z_{y}^{2}(x,y)}dxdy}$
\end{lstlisting}
%根据需要选择是否添加摘要	
%\centerline{\textbf{\zihao{4} {\heiti 摘要}}}%四号、黑体、居中的文章摘要
	
%根据需要选择是否添加关键词
%\textbf{关键词}%粗体显示的文章关键词
	
%根据需要选择是否添加目录
%\begin{文章目录}
%\clearpage
%\begin{center} 
%\tableofcontents%居中显示的文章目录
%\end{center}
%\end{文章目录}
	
\end{document}
%\end{正文结束}
