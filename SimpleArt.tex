\documentclass[12pt, a4paper, oneside]{ctexart} %文档类型是ctexart,设置字号12pt、A4纸张大小、单面打印
\usepackage{amsmath, amsthm, amssymb, appendix, bm, graphicx, hyperref, mathrsfs} % 使用的宏包

\title{\textbf{论文标题}}
\author{Sijie Li}
\date{\today}
\linespread{1.5} % 设置行间距为1.5

%%%%%%%%%%%%%%%%%%%%%定理环境%%%%%%%%%%%%%%%%%%%%%%%%
% Based on amsthm宏包
% [theorem]的作用是让所有的环境同theorem的编号
\newtheorem{theorem}{定理}[section]
\newtheorem{definition}[theorem]{定义}
\newtheorem{lemma}[theorem]{引理}
\newtheorem{corollary}[theorem]{推论}
\newtheorem{example}[theorem]{例}
\newtheorem{proposition}[theorem]{命题}

\renewcommand{\abstractname}{\Large\textbf{摘要}} % 设置“摘要”这两个字的显示方式

\begin{document}

%%%%%%%%%%%%%%%%%%%%%正文首页%%%%%%%%%%%%%%%%%%%%%%%%
\maketitle

\setcounter{page}{0} %首页页码为0
\maketitle
\thispagestyle{empty}

\begin{abstract}
    这里是摘要.
    \par\textbf{关键词:}这里是关键词; 这里是关键词.
\end{abstract}

% 目录页
\newpage 
\pagenumbering{Roman}
\setcounter{page}{1}
\tableofcontents

% 正文页
\newpage
\setcounter{page}{1}
\pagenumbering{arabic}

\section{一级标题}

\subsection{二级标题}

这里是正文.

% 参考文献
\newpage

\begin{thebibliography}{99}
    \bibitem{a}作者. \emph{文献}[M]. 地点:出版社,年份.
    \bibitem{b}作者. \emph{文献}[M]. 地点:出版社,年份.
\end{thebibliography}

% 附录
\newpage

\begin{appendices}
    \renewcommand{\thesection}{\Alph{section}}
    \section{附录标题}
        这里是附录.
\end{appendices}

\end{document}