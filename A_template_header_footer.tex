\documentclass[11pt,a4paper,oneside,UTF8,fontset=none]{ctexart}
\linespread{1.25}  % 行距
%[]为可选参数,{}为必要参数
%[]:字体:10pt,11pt,12pt
%    纸张:a4paper-a6paper以及b和c,letteraper
%    单双面:oneside,twoside
%    字体为none为后续做准备
%{}:文章类型:英文:Article,report,book
%             中英文:ctexart,ctexrep,ctexbook
%----------------------------------------------版面
\usepackage{geometry}  % 宏包
\geometry{left=1.8cm,right=1.8cm,top=2cm,bottom=2cm}
%设置页边距
\geometry{centering}
%版心水平垂直均居中,竖直居中:vcentering,水平居中:hcentering
%----------------------------------------------页眉页脚
\usepackage{fancyhdr}  % 宏包
\usepackage{lastpage}  % 获得总页数宏包
\pagestyle{fancy}
\fancyhf{}  % 清除页眉页脚保证默认样式不起作用
\lhead{lefthead}
\chead{hollow world}
\rhead{righthead}
\lfoot{通讯作者Email}
\cfoot{\thepage ;\thepage/\pageref{LastPage}}  % 编辑页码;当前页/总页数
\rfoot{Rightfoot}
\renewcommand{\headrulewidth}{0pt}  % 设置页眉线粗
\renewcommand{\footrulewidth}{0pt}  % 设置页脚线粗
\pagestyle{empty}  % 首页之后清空格式
%设置页眉页脚
%\lhead{}, \chead{}, \rhead{}, \lfoot{}, \lhead{}, \rhead{}
%----------------------------------------------字体字号
\usepackage{fontspec}  % 宏包
%题目
\newfontfamily\timuen{Times New Roman}   % 设置英文局部字体
\newCJKfontfamily\timucn{simhei.ttf}  % 设置中文局部字体 黑体
\newcommand{\timu}[1]{\timucn \timuen #1}  % 混合中英文
%作者:
\newfontfamily\zzen{Times New Roman}  % 设置英文局部字体
\newCJKfontfamily\zzcn{simfang.ttf}  % 设置中文局部字体 仿宋
\newcommand{\zz}[1]{\zzcn \zzen #1}  % 混合中英文
%单位:
\newfontfamily\dwen{Times New Roman}  % 设置英文局部字体
\newCJKfontfamily\dwcn{simsun.ttc}  % 设置中文局部字体 宋体
\newcommand{\dw}[1]{\dwcn \dwen #1}  % 混合中英文
%摘要和关键字:
\newfontfamily\zygjzten{Times New Roman}  % 设置英文局部字体
\newCJKfontfamily\zygjztcn{simhei.ttf}  % 设置中文局部字体 黑体
\newcommand{\zygjzt}[1]{\zygjztcn \zygjzten #1}  % 混合中英文
%摘要和关键字内容:
\newfontfamily\zygjzen{Times New Roman}  % 设置英文局部字体
\newCJKfontfamily\zygjzcn{simsun.ttc}  % 设置中文局部字体 宋体
\newcommand{\zygjz}[1]{\zygjzcn \zygjzen #1}  % 混合中英文
%正文内容:
\newfontfamily\zwen{Times New Roman}  % 设置英文局部字体
\newCJKfontfamily\zwcn{simsun.ttc}  % 设置中文局部字体 宋体
\newcommand{\zw}[1]{\zwcn \zwen #1}  % 混合中英文
%\setmainfont{Times New Roman}
%setCJKmainfont{SourceHanSerifSC-SemiBold.otf}
%更改全局英文字体为Times New Roman,更改中文字题为SourceHanSerifSC-SemiBold
%\newcommand{\kt}{\setCJKfamilyfont{simkai.ttf}}
%----------------------------------------------标题作者机构
\title{\Large\timu 标题}  % 输入空格为不显示 设置头{\三号字\}
	\author{\large\zz 作者$^*_{fn}$,二作\textsuperscript{上标} \\\dw\footnotesize 机构}  % 输入空格为不显示,$^$上标,$_$下标。
	\date{}  % \normalsize 日期,输入空格为不显示
	%\tiny:七号5.25pt;\scriptsize:六号7.875pt;\footnotesize:小五号9pt;
	%\small:五号10.5pt;\normalsize:小四号12pt;\large:四号13.7pt;\Large:三号15.75pt;
	%\LARGE:二号21pt;\huge:一号27.5pt;\Huge:小初号36pt。
	%----------------------------------------------各级标题
	\ctexset{
		section={
			format=\timu\raggedright\normalsize\large  % 设置 section 字体timu、左对齐(右对齐raggedlight)、无缩进、四号字
		},
		subsection={
			format=\timu\raggedright\normalsize\small  % 设置 section 字体timu、左对齐(右对齐raggedlight)、无缩进、五号字
		},
		subsubsection={
			format=\timu\raggedright\normalsize\small  % 设置 section 字体timu、左对齐(右对齐raggedlight)、无缩进、五号字
		},
	}
	%----------------------------------------------表格
	\usepackage{booktabs}  % 宏包
	\usepackage{multirow}  % 复杂表格宏包
	% 并列
	\usepackage{floatrow}
	\floatsetup[table]{capposition=top}
	\newfloatcommand{capbtabbox}{table}[][\FBwidth]
	% 表标题字号设置
	\usepackage{caption}
	\captionsetup[table]{font=footnotesize}
	%----------------------------------------------
	\begin{document}
		\maketitle  %显示标题、作者、日期
		\thispagestyle{fancy}  %显示页眉页脚
		\zygjzt\footnotesize  % 调用字体并设置字号
		\noindent{\textbf{摘\hspace{1em}要:}}
		{en-zhaiyao}\vspace{8em}
		%\不缩进{\文本加粗{摘\hspace{ }要:}摘要内容}
		%\hspace{}字符间空格,以em、cm等可以作为单位,精确控制空格长度
		%\vspace{行间距微调}
		\newline
		%换行,分段\par,分页\newpage或\clearpage
		\zygjzt\footnotesize
		\noindent{关键字:}  % \不缩进{\文本加粗{关键字:}}
		{LaTeX}\vspace{1em}
		\newline
		%换行,分段\par,分页\newpage或\clearpage
		%----------------------------------------------
		
		
		\section{我是一级标题}
		\zw\small
		分\hfill 栏\hfill 符
		
		\subsection{我是二级标题}
		分\hfill 栏\hfill 符
		\newline
		%换行,分段\par,分页\newpage或\clearpage
		疯\hfill 狂\hfill 分\hfill 栏
		%\hfill可以理解为分栏符
		
		\subsubsection{我是三级标题\ 表1}
		我就是不分栏
		\par
		疯\hfill 狂\hfill 分\hfill 栏  % 换行和分段的悬挂不一样
		%----------------------------------------------
		\begin{table}[H]  % [H]表格紧跟正文
			\begin{floatrow}
				\capbtabbox{
					\zz\scriptsize
					\begin{tabular}{ccccc}  % 需要5列
						\toprule  % 第一条线
						A                     & B                     & C                     & D                     & E                     \\
						\midrule  % 第二条线
						\multicolumn{1}{l}{1} & \multicolumn{1}{l}{2} & \multicolumn{1}{l}{3} & \multicolumn{1}{l}{4}& \multicolumn{1}{l}{5}\\  %\multicolumn{1}{l}{1}占用{1列}{居左}{内容为1}
						\bottomrule  % 第三条线
				\end{tabular}}
				{\zygjz\caption{示例1.1}\label{table4}}  % 前表标题五号字,设置一个另一个会联动,\label规定其为表1,不设置会顺序编号
				\hspace{10em}  % 增加并列间距
				\capbtabbox{
					\zz\scriptsize
					\begin{tabular}{ccccc}  % 需要5列
						\toprule  % 第一条线
						\multirow{2}{*}{A} & \multicolumn{2}{l}{B} & \multicolumn{2}{l}{C} \\  % \multirow占用{2列}{自适应宽度}{内容为A}
						& \multicolumn{1}{l}{D} & \multicolumn{1}{l}{D} & \multicolumn{1}{l}{D} & \multicolumn{1}{l}{D} \\
						\midrule  % 第二条线
						\multicolumn{1}{l}{1} & \multicolumn{1}{l}{2} & \multicolumn{1}{l}{3} & \multicolumn{1}{l}{4}& \multicolumn{1}{l}{5}\\  %\multicolumn{1}{l}{1}占用{1列}{居左}{内容为1}
						\bottomrule  % 第三条线
				\end{tabular}}
				{\zygjz\caption{示例1.2}\label{table5}}  % 后标标题五号字
			\end{floatrow}
		\end{table}
		
		\begin{table}[H]
			\centering
			{\zygjz\caption{单独示例}}
			\zz\scriptsize
			\begin{tabular}{ccccc}  % 需要5列
				\toprule  % 第一条线
				\multirow{2}{*}{A} & \multicolumn{2}{l}{B} & \multicolumn{2}{l}{C} \\  % \multirow占用{2列}{自适应宽度}{内容为A}
				& \multicolumn{1}{l}{D} & \multicolumn{1}{l}{D} & \multicolumn{1}{l}{D} & \multicolumn{1}{l}{D} \\
				\midrule  % 第二条线
				\multicolumn{1}{l}{1} & \multicolumn{1}{l}{2} & \multicolumn{1}{l}{3} & \multicolumn{1}{l}{4}& \multicolumn{1}{l}{5}\\  %\multicolumn{1}{l}{1}占用{1列}{居左}{内容为1}
				\bottomrule  % 第三条线
			\end{tabular}
		\end{table}
	\end{document}