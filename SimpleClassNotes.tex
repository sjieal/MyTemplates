\documentclass[12pt, a4paper, oneside]{ctexbook} % 文档类型是ctexbook,设置字号12pt、A4纸张大小、单面打印
\usepackage{amsmath, amsthm, amssymb, bm, graphicx, hyperref, mathrsfs} % 使用的宏包

\title{{\Huge{\textbf{笔记本标题}}}\\——副标题}
\author{Sijie Li}
\date{\today}
\linespread{1.5} % 设置行间距为1.5

%%%%%%%%%%%%%%%%%%%%%定理环境%%%%%%%%%%%%%%%%%%%%%%%%
% Based on amsthm宏包
% [theorem]的作用是让所有的环境同theorem的编号
\newtheorem{theorem}{定理}[section]
\newtheorem{definition}[theorem]{定义}
\newtheorem{lemma}[theorem]{引理}
\newtheorem{corollary}[theorem]{推论}
\newtheorem{example}[theorem]{例}
\newtheorem{proposition}[theorem]{命题}

\begin{document}

% 封面:文档类型是ctexbook的话,就很容易制作封面
\maketitle

% 加入了单独一页,用来写前言
\pagenumbering{roman}
\setcounter{page}{1}

\begin{center}
    \Huge\textbf{前言}
\end{center}~\

这是笔记的前言部分. 

\begin{flushright}
    \begin{tabular}{c}
        Sijie Li\\
        \today
    \end{tabular}
\end{flushright} % flushright有居右对齐的作用,并且打印了前言的标题。

% 目录
\newpage
\pagenumbering{Roman}
\setcounter{page}{1}
\tableofcontents

% 正文
\newpage
\setcounter{page}{1}
\pagenumbering{arabic}

\chapter{章节标题}

在这里可以输入笔记的内容.

\section{小节标题}

这是笔记的正文部分.

\end{document}