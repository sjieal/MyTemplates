\documentclass[10pt,UTF8]{ctexart}     %汉化文章
%\usepackage[UTF8]{ctex}     %采用utf8编码
%\usepackage{amsmath}        %数学公式
\usepackage{multirow}  %表格实现跨多行
%导言区的标题,作者,时间
\title{LaTeX的文章}
\author{Pacy}
\date{\today}
%宏包,可以实现目录跳转等
\usepackage{hyperref}
%定义摘要
\usepackage{abstract}  %先加载包
\renewcommand{\abstractname}{\Large{\textbf{摘要}}}
\usepackage{ctex}	
\usepackage{xcolor}   %调用颜色的宏包
\begin{document}
	\maketitle  %显示导言区标题
	\setcounter{page}{0}   %设置页码
	\thispagestyle{empty}  %只改变这一页,取消页码显示
	\maketitle  %显示摘要
	%写摘要内容
	\begin{abstract}
		主要是写了LaTex的一些基础排版知识,可以嵌套里面格式排版一些比较简单的文章,今天就学到这里啦!!! \par  %\par是一个换行命令,且首行自动缩进
		%以下写关键词
		\textbf{关键词:}Latex、初学、快速入门   %\textbf显示黑体
	\end{abstract}
	%进入下一页
	\newpage
	%设置目录
	\tableofcontents
	\pagenumbering{Roman}  %采用罗马数字编号
	\setcounter{page}{1}  %从1开始编号
	%下一页
	\newpage
	%开始写正文
	%先设置页码格式
	\pagenumbering{arabic}  %阿拉伯数字编号
	\setcounter{page}{1}  %从1开始
	%设置第一节
	\section{相关符号}
	%设置第一小节
	\subsection{中英文左右引号}
	1、英文引号\par
	``这是左双引号  \par %`为上排数字1左侧,英文输入下
	''这是右双引号 \par  %'为enter左侧
	2、中文左右双引号可以在中文输入法下输出 \par
	“我是左和右引号”  \par
	%第二小节
	\subsection{短横、省略号、破折号}
	1、英文的短横-   \par
	连字符(一个短-):daughter-in-law \par
	数字起止符(两个短--):page 1--2  \par
	破折号(三个短---):Listen---I'm serious \par
	2、中文  \par
	中文破折号——,在中文输入法下输入即可\par
	中文省略号...,在英文下输入三个.即可\par
	而英文下的省略号,要用\texttt{\char`\\}dots显示$\dots$。\par  %$ $是行内公式输入的一个环境,\dots要放在里面才能显示
	%反斜杠\的输出比较特殊,可以用\texttt{\char`\\}输出反斜杠
	
	%下一节
	\section{字体}
	%第一小节
	\subsection{字族}
	%中文粗体是黑体表示,斜体是楷书表示
	\songti 宋体格式——好好感受一下下不同字体哈哈哈哈哈哈哈哈哈哈哈哈,啦啦啦啦。 \par
	\heiti 黑体格式——好好感受一下下不同字体哈哈哈哈哈哈哈哈哈哈哈哈,啦啦啦啦。 \par
	\fangsong 仿宋格式——好好感受一下下不同字体哈哈哈哈哈哈哈哈哈哈哈哈,啦啦啦啦。\par
	\kaishu 楷书格式——好好感受一下下不同字体哈哈哈哈哈哈哈哈哈哈哈哈,啦啦啦啦。\par
	
	
	\subsection{斜体、粗体}
	%斜体内容
	\textsl{现在使用的是将变为伪斜体。斜体Slant字形。可以看英文比较明显一点。} \par
	\textit{现在使用的是变为斜体。斜体Italic字形。}\par
	%粗体
	\textbf{对文字进行一个加粗处理。粗体BoldSeries。} \par
	\textmd{中粗体。MiddleSeries。} \par
	
	\subsection{字号}
	%表示一号到八号字体
	
	\zihao1 一号字体 \par
	\zihao2 二号字体 \par
	\zihao3 三号字体 \par
	\zihao4 四号字体 \par
	\zihao5 五号字体 \par
	\zihao6 六号字体 \par
	\zihao7 七号字体 \par
	\zihao8 八号字体 \par
	%表示小初号到小六号字体
	\zihao{-1} 小一号字体 \par
	\zihao{-2} 小二号字体 \par
	\zihao{-3} 小三号字体 \par
	\zihao{-4} 小四号字体 \par
	\zihao{-5} 小五号字体 \par
	\zihao{-6} 小六号字体 \par
	%日常用的小四号为12pt,五号为10.5pt
	%如果想要设置特殊字号,如下,10.5指字体大小10.5pt;\baselineskip指line-height行距,1.2倍行高;默认字体\rmfamily
	\fontsize{12}{\baselineskip}{\rmfamily}
	设置好的字体就是这样子了。\par
	\zihao{4}
	\subsection{颜色}
	\definecolor{color1}{RGB}{250,34,20}  %定义颜色名color1的颜色
	{\color{color1}{text}}   %设置为color1的颜色
	{\color{blue}{text}}   %设置为蓝色
	{\color{purple}{text}}   %设置为color1的颜色紫色
	
	\section{3}
	\subsection{标签、引用}
	%\label{key} 插入标签,\pageref{label}或者\ref{label}进行引用
	
	\begin{equation}\label{1}  %标号记为1
	F=ma
	\end{equation}
	\begin{equation}\label{2} 
	v=v_0 t
	\end{equation}
	
	接下来我们来写一下引用,由(\ref{1})(\ref{2})得:得到有关运动学的一些结论。

	\subsection{脚注}
	%括号内写的是脚注的内容
	这是一个脚注地方,太阳花\footnote{太阳花是一种花}
	
	\subsection{列表}
	%无序列表
	无序列表 
	\begin{itemize}
		\item 苹果好吃吗?
		\item 香蕉好吃吗?
		\item 西瓜好吃吗?
	\end{itemize}
	
	%有序列表
	有序列表
	\begin{enumerate}
		\item 昨天跑步了
		\item 今天跑步了
		\item 明天要跑步
	\end{enumerate}
	%改编号
		有序列表改编号
	\begin{enumerate}
		\item[a、] 昨天跑步了
		\item[b、] 今天跑步了
		\item[c、] 明天要跑步
	\end{enumerate}
	%描述列表
	
	描述型列表 %方括号内容会加粗显示
	\begin{description}
		\item[LaTeX:] 一个排版系统
		\item[.tex:] LaTeX文档扩展名
	\end{description}
	
	\subsection{表格}
	\begin{center}  %表格居中
		\begin{tabular}[t]{|l|c||p{3em}r@{-}}
		%可选参数[t]表示表格上端与同一行文字上端同高
		%参数[c]表示与文字中央同高
		%参数[b]表示文字与表格下端同高
		%必选参数{}中|表示表格竖线及个数,最右边没|,表示留空,或者用@{}表示留空,也可以@{-}把竖直线换成-。
		%lcr分别表示左对齐,居中,右对齐
		%p{3em}表示指定某一列宽度,文字自动左对齐
		%\hline表示绘制水平线
		
			\hline\hline   %所有列画水平线,两条
			A&B&C&d\\      %&用于分割每一列
			D&E&F&g\\
			\cline{1-2}   %只在12列画横线
			\multicolumn{2}{|c|}{G}&H&i\\  %\multicolumn{cols}{pos}{text}跨列命令
			\hline
		\end{tabular}
	\end{center}

	
\end{document}