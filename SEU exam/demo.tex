% !TEX encoding = UTF-8
% !TEX program = latexmk
\documentclass{SEUExam}
% =================================================
%       PDF信息
% =================================================
% PDF信息里的作者栏
\author{Southeast University·Jieal}
% PDF信息里的主题
\Subject{Examination for SEU}
% PDF信息里的关键词
\Keywords{SEM; Logistics; 148201; ERP}

% =================================================
%       试卷头信息
% =================================================
% 年份
\Year{2023}
% 学期
\Semester{1}
% 课程
\CourseID{B1410310}
\Course{企业资源规划(研讨)}

% 类型,如A/B/模拟等
\Type{A}
% 适用专业
\Major{物流管理,电子商务}
% 考试形式
\Format{闭\hspace*{1em}卷}
% 考试时长
\Duration{120分钟}
% 计分表中大题的数目
\TotalPart{5}

\examsetup{
  % page/size = letter,
}
%\renewcommand{\baselinestretch}{1.5}

\begin{document}
\maketitle

\section{填空题(本题共10小题,每题4分,满分40分)}

1. $\lim _{x \to 0} \frac{\sin (\sin (\sin x))}{\tan x}=$
  \fillin[ ].


\section{单项选择题(本题共10小题,每题4分,满分40分)}

  1. 设当 $x \in(0,+\infty)$ 时, $f(x)=x \sin \frac{1}{x}$, 则在 $(0,+\infty)$ 内
  \paren[ ]
  \begin{choices}
    \item $f(x)$ 与 $f^{\prime}(x)$ 都无界
    \item $f(x)$ 有界, $f^{\prime}(x)$ 无界
    \item $f(x)$ 与 $f^{\prime}(x)$ 都有界
    \item $f(x)$ 无界, $f^{\prime}(x)$ 有界
  \end{choices}

  2. 管理活动通常分为高、中、基三个层次,分别对应三种类型决策过程,分别是 \paren[ ]
  \begin{choices}
    \item 非结构化决策、半结构化决策和结构化决策
    \item 非结构化决策、结构化决策和半结构化决策
    \item 主观性决策、一般性决策和过程性决策
    \item 主观性决策、过程性决策和一般性决策
  \end{choices}
  
  3. 下列说法不正确的是 \paren[ ]
  \begin{choices}
    \item 计算机网络是信息共享的基础
    \item 管理信息系统是一个技术系统
    \item 数据库为管理信息系统提供了信息的战略储备和供给
    \item 现代化的管理组织和协调为管理信息系统提供了一颗奔腾的芯
  \end{choices}
  
\section{计算题(本题共2小题,每题6分,满分12分)}

1. $\lim _{n \to \infty} n\left[\left(1+\frac{1}{n}\right)^{n}-\upe\right]$

\section{证明题(8分)}

1. 设 $f(x)$ 在 $[a, b]$ 上连续, 在 $(a, b)$ 内可导, 试证至少存在一点 $\xi$, 使得
  $$f(b)-f(a)=\xi f^{\prime}(\xi)(\ln b-\ln a).$$

\begin{proof}[show-qed=false]
  由柯西中值定理可知, 当 $f(x)$ 在 $[a, b]$ 上连续, 在 $(a, b)$ 内可导,

  有
  $\frac{f(b)-f(a)}{g(b)-g(a)}=\frac{f^{\prime}(\xi)}{g^{\prime}(\xi)}$,
  即
  $$\frac{f(b)-f(a)}{\ln b-\ln a}=\frac{f^{\prime}(\xi)}{\frac{1}{\xi}}
    ~\Rightarrow~
    f(b)-f(a)=\xi f^{\prime}(\xi)(\ln b-\ln a) $$
  证毕.
\end{proof}

\end{document}
